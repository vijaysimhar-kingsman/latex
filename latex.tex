\documentclass[13pt , a4paper]{article}
\usepackage[utf8]{inputenc}
\usepackage[left=0.8in , right=0.7in , top=1in , bottom=0.8in]{geometry}
\usepackage{fancyhdr}
\usepackage{enumitem}
\usepackage{graphicx}
\usepackage{fancybox}
\usepackage{booktabs}
\usepackage{tgbonum}
\usepackage[T1]{fontenc}
\usepackage{lineno, blindtext}
\usepackage{natbib}
\usepackage[switch, modulo]{lineno}
\usepackage{lipsum}
\setlength{\parindent}{0em}
\def \v{\vskip3mm}
\textwidth 45em


\begin{document}
\begin{center}
\Large{UNIVERSITY OF MARS \\
INSTITUTE OF INTERGALACTIC TRAVEL}

\vskip3mm
\hrule
\v
\Huge{\textbf{Homework \# 1}}
\v
\small{student name : \textit{vijayasimha Reddy}}
\v
\hrule
\v
Course:\textit{ Special Relativity (Physics 301) – Professor: Dr. Albert Einstein \\
Due date: March 28th, 2025}
\vskip15mm
\Large{\textbf{Question 1}}
\vskip8mm
\fbox{\begin{minipage}{31em}
What is the airspeed velocity of an unladen swallow?
\end{minipage}}
\v
\includegraphics[scale=1.5]{birdu.png}
\v
\end{center}
\begin{addmargin}
\textbf{Answer.} While this question leaves out the crucial element of the geographic origin of
the swallow, according to Jonathan Corum, an unladen European swallow maintains
a cruising airspeed velocity of 11 metres per second, or 24 miles an hour. The velocity
of the corresponding African swallows requires further research as kinematic data is
severely lacking for these species.
\end{addmargin}
\begin{center}
    \Large{\textbf{Question 2}}
    \vskip8mm
\fbox{\begin{minipage}{31em}
\normalsize How much wood would a woodchuck chuck if a woodchuck could chuck wood?\\
\normalsize\textit{(a) Suppose “chuck" implies throwing.}\\
\normasize\textit{(b) Suppose “chuck" implies vomiting.}\\
\end{minipage}}
\end{center}
\textbf{Answer.}\\
(a)
\begin{enumerate}
\vskip-6mm
According to the Associated Press (1988), a New York Fish and Wildlife technician
named Richard Thomas calculated the volume of dirt in a typical 25–30 foot (7.6–
9.1 m) long woodchuck burrow and had determined that if the woodchuck had
moved an equivalent volume of wood, it could move “about\textbf{ 700 pounds (320 kg)}
on a good day, with the wind at his back".
\end{enumerate}
\newpage
\vskip5mm
\textit{Special Relativity (Physics 301)}– Homework \#1
\vskip1mm
\hrule
\begin{flushright}
\vskip-0.7cm
2
\end{flushright}
\vskip -1cm
\vskip12mm
(b)\vskip-6.5mm
\begin{enumerate}
A woodchuck can ingest $361.92 cm^3$ \(22.09 cu in\) of wood per day. Assuming immediate expulsion on ingestion with a 5\% retainment rate, a woodchuck could chuck \textbf{$343.82 cm^3$}  of wood per day.
\end{enumerate}
\begin{center}
    \vskip10mm
    \Large{\textbf{Question 3}}
    \vskip8mm
    \fbox{\begin{minipage}{31em}
\normalsize Identify the author of Equation 1 below and briefly describe it in Latin.
\normalsize$$ P(A|B)=\frac{P(B|A)P(A)}{P(B)}$$\modulolinenumbers[3]
\end{minipage}}
\end{center}
\vskip10mm
\textbf{Answer.} Lorem ipsum dolor sit amet, consectetur adipiscing elit. Praesent porttitorarcu luctus, imperdiet urna iaculis, mattis eros. Pellentesque iaculis odio vel nisl ullamcorper, nec faucibus ipsum molestie. Sed dictum nisl non aliquet porttitor. Etiavulputate arcu dignissim, finibus sem et, viverra nisl. Aenean luctus congue massa, utlaoreet metus ornare in. Nunc fermentum nisi imperdiet lectus tincidunt vestibulumat ac elit. Nulla mattis nisl eu malesuada suscipit.
\begin{center}
    \vskip5mm
    \Large{\textbf{Question 4 (bonus marks)}}
    \vskip8mm
    \fbox{\begin{minipage}{31em}\normalsize
    The table below shows the nutritional consistencies of two sausage types. Explain their relative differences given what you know about daily adult nutritional recommendations.
   \vskip5mm
     \begin{center}
         \begin{tabular}{l c r}
        \toprule
          Per50g &    Pork    & Soy\\
          \midrule
            Energy        &  760kJ   &  538kJ\\
            Protein       &  7.0g    &  9.3g\\
            Carbohydrate  &  0.0g    &  4.9g \\
            Fat           &  16.8g   &  9.1g\\
            Sodium        &  0.4g    &  0.4g\\
            Fibre         &  0.0g    &  1.4g\\
            \bottomrule
            
        \end{tabular}
       
     \end{center}      
    \end{minipage}}
\end{center}
\vskip1mm

\textbf{Answer.} Lorem ipsum dolor sit amet, consectetur adipiscing elit. Praesent porttitorarcu luctus, imperdiet urna iaculis, mattis eros. Pellentesque iaculis odio vel nisl ullamcorper, nec faucibus ipsum molestie. Sed dictum nisl non aliquet porttitor. Etiamvulputate arcu dignissim, finibus sem et, viverra nisl. Aenean luctus congue massa, utlaoreet metus ornare in. Nunc fermentum nisi imperdiet lectus tincidunt vestibulumat ac elit. Nulla mattis nisl eu malesuada suscipit.\\
\pagestyle{empty}
\vskip0.9mm
\hrule
\begin{flushright}
\textit{Felipe Portales-Oliva}
\end{flushright}
\newpage
\vskip5mm
\textit{Special Relativity (Physics 301)}– Homework \#1
\vskip1mm
\hrule
\begin{flushright}
\vskip-0.7cm
3
\end{flushright}
\vskip -1cm

\begin{center}\\
\vskip3mm
\textbf{Question 5 (bonus marks)}\\
\vskip5mm
    \fbox{\begin{minipage}{46em}

        \begin{center}
          Listing 1: Luftballons Perl Script.
          \vskip-2mm
        \end{center}
        
\end{minipage}}\\

\vskip-0.7mm

    \fbox{\begin{minipage}{46em}
    \begin{linenumbers}     
         {\fontfamily{qcr}\selectfont
         \text \#!/usr/bin/perl}
        \textbf \\{use} strict;\\
        \textbf{use} warnings ;\\
        \\for(1..99 ) \{ \textbf{print} \$_."Luftballons \,\textbackslash n" ; \}\\
         \\{\fontfamily{qcr}\selectfont
        \text \# This is a commented line}  \\
         \\\textbf{my} \$ s t ri n g = " Hello World ! " ;\\
         \\\textbf{print} \$ string . "\textbackslash n\textbackslash n";\\
         \\\$ string =\sim s/Hello/Goodbye Cruel /;\\
        \\\textbf{print} \$ string . "\textbackslash n\textbackslash n";\\
        \\finale();\\
        \\exit;\\
        \\sub finale\{ \textbf{print} " Fin .\textbackslash n " ; \}\\
     \end{linenumbers}  
    \end{minipage}}
    \vskip-0.5mm
     \fbox{\begin{minipage}{46em}
     \begin{enumerate}
         \item  How many luftballons will be output by the Listing 1 above?
\item Identify the regular expression in Listing 1 and explain how it relates to the anti-war sentiments found in the rest of the script.\\
     \end{enumerate}
    \end{minipage}}\\ 
\end{center}\\
 \textbf{Answer .}\\
\begin{enumerate}
    \item  99 luftballons.
\item Lorem ipsum dolor sit amet, consectetur adipiscing elit. Praesent porttitor arcu
luctus, imperdiet urna iaculis, mattis eros. Pellentesque iaculis odio vel nislullamcorper, nec faucibus ipsum molestie. Sed dictum nisl non aliquet portti-
tor. Etiam vulputate arcu dignissim, finibus sem et, viverra nisl. Aenean luctus
congue massa, ut laoreet metus ornare in. Nunc fermentum nisi imperdiet lectus
tincidunt vestibulum at ac elit. Nulla mattis nisl eu malesuada suscipit.
\end{enumerate}
 \vskip6.2cm

\hrule
\begin{flushright}
\textit{Felipe Portales-Oliva}

\end{flushright}
\pagestyle{empty}
\end{document}